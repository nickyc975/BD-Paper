\documentclass{elegantpaper}

\title{分布式空间top-k关键字查询}

\author{
    冯运 1160300524 %
    \\[0.5ex] %
    陈进龙 1160300525 %
}

\begin{document}

\maketitle

\begin{abstract}
\end{abstract}

\section{问题描述}

\section{系统设计}

本节首先介绍一种分布式的索引结构,并基于此结构实现一种简单的查询算法。然后介绍基于此设计的一些改进以及相应的新的查询算法。

\subsection{分布式R树}

对于空间数据索引,首选的结构就是R树。但是传统的R树是基于单机的,没有考虑分布式计算的需求,故我们对R树做了一些改进,使其支持了分布式环境。
分布式R树的基本思想是将R树的节点与物理计算节点对应,即R树的一个节点存放在一个物理计算节点上。父节点所在的物理计算节点记录其所有子节点所包含的在的物理计算节点。

\subsubsection{内结点保存内容}

\begin{itemize}

    \item 本节点的MBR

    \item 子节点MBR集

    \item 每个子节点MBR对应的物理节点
    
\end{itemize}

\subsubsection{叶节点保存内容}

\begin{itemize}

    \item 本节点的MBR
    
    \item 在本区域出现的关键字列表,其中的项为形似\verb|<t, t.entries, t.freq>|的三元组,其中\verb|t|表示关键字,\verb|t.entries|为形似\verb|<loc, freq>|的二元组集合,表示\verb|t|出现的位置及在该位置出现的次数,\verb|t.freq|表示\verb|t|出现的总次数。

\end{itemize}

\subsection{索引创建}

在向索引集群导入数据的过程中,自动根据数据创建索引,具体过程如下:

\begin{itemize}
    
    \item[1.] 初始只有一个根节点服务器;
    
    \item[2.] 当数据量小于阈值$\delta$时,直接在根节点服务器上保存,即根服务器就作为唯一的叶节点;
    
    \item[3.] 当数据量大于等于阈值$\delta$时,向集群管理器申请新的多个新的计算节点,将数据按空间区域进行划分,并将划分后的数据分配到新申请的节点中,在本节点只保存划分信息;
    
    \item[4.] 

\end{itemize}

\subsection{执行查询}



\section{工作流程}

\end{document}